\documentclass[10pt, a4paper]{article}

\usepackage[utf8]{inputenc}
\usepackage[english, spanish]{babel}
\usepackage[left=25mm, right=25mm, top=35mm, bottom=30mm, headheight=35mm]{geometry}
\usepackage{graphicx}
\usepackage{float}
\usepackage{xcolor}
\usepackage{fancyhdr}
\usepackage{hyperref}
\usepackage{setspace}
\usepackage{indentfirst}

% Syntax customization with minted package
\usepackage{minted}
\usemintedstyle{nord}
\setminted{
  breaklines,
  linenos,
  frame=lines,
  fontsize=\normalsize
}
\newcommand{\mintpython}[1]{\mintinline[style=gruvbox-light]{python}{#1}}

% Define background color
\definecolor{background}{HTML}{2E3440}

% Variables
\newcommand{\university}{Universidad Nacional de San Agustín de Arequipa}
\newcommand{\faculty}{Facultad de Ingeniería de Producción y Servicios}
\newcommand{\program}{Escuela Profesional de Ingeniería de Sistemas}
\newcommand{\semester}{2024 - A}
\newcommand{\course}{img/web_programming.png}
\newcommand{\topic}{img/python.png}
\newcommand{\professor}{Carlo Jose Luis Corrales Delgado}
\newcommand{\student}{Jorge Luis Mamani Huarsaya}
\newcommand{\email}{https://mail.google.com/mail/u/0/?fs=1&tf=cm&source=mailto&to=jmamanihuars@unsa.edu.pe}
\newcommand{\github}{https://github.com/jorghee/chess-with-python}
\newcommand{\mydate}{24 de mayo, 2024}

% Just parts and chapters numbered
\setcounter{secnumdepth}{0}

% Head and foot customization
\pagestyle{fancy}
\lhead{\raisebox{-0.2\height}{\includegraphics[width=4cm]{img/logo_unsa.png}}}
\chead{\fontsize{8}{8}\selectfont \university \\ \faculty \\ \textbf{\program}}
\rhead{\raisebox{-0.2\height}{\includegraphics[width=3.5cm]{img/logo_episunsa.png}}}
\lfoot{Estudiante \student}
\cfoot{}
\rfoot{Pág. \thepage}

\begin{document}

\begin{titlepage}
	\centering
	\includegraphics[width=15cm]{\course} \par
  \vfill \vfill
	\includegraphics[width=15cm]{\topic}\par
  \vfill \vfill
  {\textbf{Profesor(a):} \par}
	\professor \vfill
  {\textbf{Estudiante:} \par}
	\student \vfill
  {\textbf{Email:} \par}
  \href{\email}{jmamanihuars@unsa.edu.pe} \vfill
  {\textbf{Repositorio GitHub:} \par}
  \href{\github}{\github} \vfill
	{\large \mydate \par}
\end{titlepage}

\section{Implementando los métodos de la clase Picture}
Para entender toda la lógica que se explicará a continuación debemos tener claro los siguiente conceptos:

\begin{itemize}
  \item La clase Picture tiene un campo denominado \textbf{img}, el cual es una lista de cadenas de texto (strings). Lo curioso es que esta lista de strings representa la figura que usando el paquete \textbf{pygame} y la lógica del archivo \href{https://github.com/jorghee/chess-with-python/blob/main/interpreter.py}{interpreter.py} se transforma de caracteres a pixeles.
  \item Nosotros no debemos preocuparnos por la conversión de caracteres a pixeles, nuestra aventura es implementar la manipulación de los strings, por lo tanto vamos a trabajar con la lista de strings \mintpython{self.img}
\end{itemize}

\subsection{El método \mintpython{horizontalMirror()}}
Este método devuelve el espejo horizontal de la imagen, es decir, el giro se hace alrededor de una linea imaginaria horizontal trazada en medio de la imagen. 
\singlespacing
La lógica en mente es iterar sobre la lista de string del objeto Picture al cual se le esta aplicanco este método y pasar el últimstring de la lista al primer elemento de una nueva lista que denominaremos \mintpython{horizontal}

\begin{itemize}
  \item Primero creamos la lista \mintpython{horizontal} vacía con el objetivo de que según la iteración se vayan agregando las los strings de \mintpython{self.img}
  \item Ahora declaramos la estructura de control \mintpython{for in} para iterar sobre \mintpython{self.img}. Empezamos interando desde el último elemento de esta lista y dicho valor lo asignamos al primer elemento de la lista creada inicialmente vacía.
\end{itemize}

\begin{minted}[bgcolor=background]{python}
# Iterando desde el último elemento de la listas hacia el primer elmemento
def horizontalMirror(self):
  size = len(self.img)
  horizontal = []
  for value in range(size):
    horizontal.append(self.img[size - 1 - value])
  return Picture(horizontal)
\end{minted}

\subsection{}
\subsection{}
\subsection{}
\subsection{}
\subsection{}
\subsection{}
\subsection{}
\subsection{}
\subsection{}
\section{Resolución de los ejercicios}
\subsection{}
\subsection{}
\subsection{}
\subsection{}
\subsection{}
\subsection{}
\subsection{}

\end{document}
